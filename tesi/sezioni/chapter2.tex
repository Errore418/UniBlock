\myChapter{Concetti essenziali sulla blockchain}
Nella sua pi� primitiva definizione una blockchain � una struttura dati in grado di registrare informazioni e garantirne l'immutabilit� nel tempo, ridondando i dati su un gran numero di nodi senza che sia necessaria un'entit� centrale che vigili su possibili minacce o malfunzionamenti. Concepita nel 2008 da Satoshi Nakamoto \cite{satoshi} come mezzo per decentralizzare il mondo delle transazioni finanziarie, � stata oggetto nel corso del tempo di un forte interesse accademico e commerciale venendo applicata e sviluppata in un gran numero di scenari diversi. Pur essendo presenti sul panorama attuale svariati esempi di blockchain anche molto diverse tra loro in complessit� e implementazione, possiamo individuare alcune primitive di base comuni a tutte.

% TODO sezione che introduca in generale i meccanismi alla base della blockchain, funzioni hash, strutture dati a catena, merkle tree, ecc., focalizzandosi su quelle cose che lei ha considerato nella sua implementazione di Uniblock

\section{Blocchi e transazioni}
In una blockchain le informazioni sono tipicamente registrate in transazioni, a loro volta raggruppato in blocchi. Alla

\section{Modalit� di accesso alla rete}
Una prima differenza fondamentale delle blockchain � la modalit� con cui i nuovi nodi possono entrare a far parte della rete. Nelle blockchain pubbliche qualunque dispositivo pu� diventare un nodo della rete senza alcun controllo sulla sua legittimit�. Per questo motivo tali blockchain prendono il nome di permissionless blockchain in quanto non c'� alcuna differenza gerarchica tra i vari nodi e chiunque pu� entrare a farne parte. In generale questo tipo di blockchain presenta un numero di partecipanti molto elevato grazie alla bassa soglia di entrata. Nelle blockchain private invece l'entrata nella rete � preceduta da una fase di autenticazione del soggetto come l'appartenenza a una determinata azienda. Tali blockchain sono quindi confinate nelle realt� che le sviluppano e le loro pool di nodi sono di conseguenza molto ridotte in quanto solo chi � qualificato pu� entrare a farne parte. Inoltre riflettendo la gerarchia dell'organizzazione proprietaria della blockchain sono in genere presenti differenze di responsabilit� tra i vari nodi, da qui il nome di permissioned blockchain. Possiamo identificare inoltre un terzo tipo di blockchain, quello ibrido, in cui vengono uniti alcune caratteristiche delle blockchain pubbliche con quelle private. Nelle blockchain ibride alcune funzioni sono lasciate come pubbliche mentre altre richiedono una preventiva autenticazione. Le blockchain pubbliche sono di gran lunga il tipo pi� comune di blockchain, ma quelle private stanno guadagnando terreno avendo attirato l'attenzione del mondo finanziario \cite{HELLIAR2020102136}.

\section{Algoritmo di consenso}
Data la natura distribuita di una blockchain � di fondamentale importanza la ricerca del consenso tra i nodi, ovvero che ogni transazione generata venga validata e diffusa attraverso la rete rimanendo inalterata. Il primo algoritmo di consenso concepito � la proof of work: l'immutabilit� dei dati � garantita dalla difficolt� di computare rapidamente un puzzle crittografico. Nella rete Bitcoin ad esempio un nuovo blocco � considerato valido quando viene trovato un numero tale che inserito nell'header del blocco stesso rende il suo hash inferiore a una cert� quantit�. Variando tale quantit� � possibile modulare il carico dei blocchi generati dall'intera rete, dando il tempo ai blocchi di diffondersi tra i vari nodi. Ogni nodo d� la sua fiducia alla catena di blocchi in cui � stata spesa la maggior quantit� di lavoro computazionale. Un algoritmo di consenso concepito pi� recentemente � la proof of stake: i nuovi blocchi vengono validati da nodi scelti casualmente in base a quanta valuta hanno investito nella rete. Pur ritenendo un certo livello di casualit�, l'algoritmo di selezione del prossimo validatore privilegia i maggiori scommettitori. La legittimit� della rete � garantita perci� dal fatto che chi ha investito maggiormente avr� interesse nel suo corretto funzionamento. La rete Ethereum prevede di effettuare il cambio da proof of work a proof of stake nei prossimi anni. Per un'analisi dettagliata sugli algoritmi di consenso fare riferimento a \cite{8632190}.

\section{Privacy e sicurezza}
Nelle blockchain pubbliche il contenuto dei blocchi � disponibile a ogni nodo senza alcuna protezione. Assume grande importanza quindi avere sempre ben presente quali informazioni si stanno immettendo nella rete in chiaro e quali invece si cerca di proteggere. Nella rete Bitcoin tutto il contenuto di un blocco � presente completamente in chiaro tanto da poter ricostruire la storia di ogni singolo bitcoin fino al momento della sua coniazione. In questo caso la privacy offerta agli utenti si limita alla loro anonimizzazione nella rete offrendo indirizzi usa e getta senza legami con l'identit� legale del soggetto che li possiede. Anche la rete Ethereum non prevede di base alcuna forma di cifrazione del contenuto dei blocchi, essendo tuttavia presenti delle forme di privacy a zero conoscienza implementabili come zk-SNARKS per nascondere integralmente il contenuto di una transazione \cite{snark}. Discorso diverso invece per le blockchain private in quanto la preventiva autenticazione dei nodi permette di garantire che le informazioni contenute nei blocchi siano consultabili solo da soggetti autorizzati, non rendendo necessari complessi sistemi di crittografia per garantire la riservatezza delle informazioni.

\section{Confronto di alcune blockchain}
Di seguito una comparazione di tre grandi blockchain moderne con UniBlock:
\newline
\newline
\centerline{
\begin{tabular}{ |c||c|c|c|c| } 
 \hline
 & \large{\textit{\textbf{Bitcoin}}} & \large{\textit{\textbf{Ethereum}}} & \large{\textit{\textbf{Hyperledger Fabric}}} & \large{\textit{\textbf{UniBlock}}} \\ 
 \hline
 \hline
 \large{\textit{\textbf{Accesso}}} & Pubblica & Pubblica & Privata & Ibrida \\ 
 \hline
 \large{\textit{\textbf{Consenso}}} & Proof of work & Proof of work & Crash Fault Tolerance & Proof of work \\ 
 \hline
 \large{\textit{\textbf{Privacy}}} & Non crittografata & Non crittografata & Non crittografata & Crittografata \\
 \hline
\end{tabular}
}
